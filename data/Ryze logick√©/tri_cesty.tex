Tři cesty
85 %

%TEXT%
\footnote{Hádanka z~Pikomatu. Originální znění naleznete na http://pikomat.mff.cuni.cz, 24. ročník, 1. série, 7. úloha. Tam naleznete i řešení, pokud to mé nepochopíte.}Přijdete na rozcestí ze kterého vedou tři cesty. Pouze jediná je správná - vede k~životu. U~každé cesty stojí muž, všichni jsou bratři. U~té úplně vpravo stojí Josef, u~té prostřední stojí Václav a u~té poslední vlevo stojí Jakub. Každý buď lže nebo mluví pravdu. Říkají postupně toto:
\begin{enumerate}
\item Josef: „Správná cesta je ta Václavova nebo Jakubova.“
\item Václav: „Správná cesta je ta Jakubova.“
\item Jakub: „Alespoň dva z~nás lžou.“
\end{enumerate}
Po chvíli ještě Josef prohlásí o~jednom ze svých bratrů, že stojí u~cesty ke smrti. Přijdete na to, která cesta vede k~životu?

%BAD%
\begin{enumerate}
\item Vetšinou člověk udělá chybu v~nějaké nepodstatné úvaze, od které se pak odvýjí chybné teorie.\footnote{Musím se pochlubit, že tohle je jedna z~mála hádanek, u~kterých jsem prokazatelně sám přišel na řešení :)}
\end{enumerate}

%HELP%
Tahle hádanka je jedna z~těch těžších, doporučuji vzít si papír a tužku a všechny si nakreslit. Každý z~nich buď lže nebo mluví pravdu. Pořád. Nemohou jednou říct pravdu a jednou lhát.

%SOLUTION%
Správná cesta je ta Jakubova. Vezmeme to podle jednotlivých prohlášení, pěkně popořadě:
\begin{enumerate}
\item Josef prohlásil, že správná cesta je ta Václavova nebo Jakubova. Takže pokud lže, je správná cesta ta jeho.
\item Václav prohlásil, že správná cesta je ta Jakubova. Pokud by lhal, mohla by být správná cesta ta Václavova nebo Josefova.
\item Jakub prohlásí, že dva z~nich lžou. Pokud mluví pravdu, musí lhát jak Josef tak Václav. Pokud Jakub lže, musí Josef i Václav mluvit pravdu. Jinak to smysl nedává.
\end{enumerate}

Takže se nám vlastně zúžil výběr; pokud Jakub lže, musel musí stát u~správné cesty on. Pokud mluví pravdu musí i správné cesty stát Josef.

Teď je důležité si uvědomit, jak je vlastně hádanka myšlená. Je řečeno, že každý z~nich buď lže nebo mluví pravdu. A~je důležité si uvědomit, že to platí napořád. Nestřídají lež s~pravdou. Proto kdyby Jakub mluvil pravdu, musel by lhát Josef i v~prohlášení, které učinil potom. Prohlásil o~jednom ze svých bratrů, že stojí u~cestě ke smrti - pokud by lhal, stál by jeho bratr u~cestě k~životu. Ale teď se podívejme zpět, kdo nám vychází u~třetího bodu, pokud mluví Jakub pravdu stojí u~správné cesty Josef. Teď nám ale z~Josefova prohlášení vychází, pokud lže, že správná cesta je u~jednoho z~jeho bratrů. A~to by pak byly dvě správné cesty, ta Josefova a ta jeho bratra. My víme, že správná je jen jedna. Proto musí Jakub lhát, Josef a Václav mluví pravdu - správná cesta je ta Jakubova. Smysl to dává i pro poslední Josefovo prohlášení - prohlásí, že cesta ke smrti je ta Václavova - tím mluví pravdu.
