Šašek a král
70 %

%TEXT%
V~jednom království žil král a šašek. Král nenáviděl šaška a šašek nenáviděl krále. Ale protože to byli duší rytíři a myslí mudrci, rozhodli se vyzvat na zvláštní souboj. Na hradě se nacházelo 12 lahviček s~jedem. Nejúčinnější byl jed s~označením 12, nejslabší byl jed s~číslem 1. Jedy to byly zvláštní, pokud jste vypili jed a pak ještě jeden silnější, účinek se vynuloval a vy jste přežili.

K~jedům první přišel král. Vzal si všechny se sudým číslem a odešel. Po něm přišel šašek a vzal zbytek, tedy jedy s~lichým číslem. Večer král šaška pozval na sklenici vína. Nenáviděli se a chtěli se otrávit. Král nalil místo vína jed šaškovi, pak šašek nalil králi. Načež si každý nalil sobě, aby se pokusil jed zneutralizovat.

Následujícího se probudil šašek. Král zemřel. Jak to šašek dokázal?


%BAD%
\begin{enumerate}
\item Ne, šašek neměl šanci sehnat silnější jed.
\item Oba chtěli přežít, král nespáchal sebevraždu.
\end{enumerate}

%HELP%
Král udělal přesně to, co by v~jeho situaci udělala většina lidí. Šašek naopak zariskoval a provedl geniální tah.

%SOLUTION%
Král nalil podle očekávání šaškovi jed č.10. Šašek ale provedl riskantní a geniální kousek a místo jedu nalil králi skutečné víno – to král netušil a otrávil se sám jedem, který si nalil (č.12) aby domnělý jed neutralizoval. Šašek pak jed zapil č.11, čímž ho zneutraliozal.
