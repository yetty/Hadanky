\documentclass[czech]{book}
\usepackage[T1]{fontenc}
\usepackage[utf8]{inputenc}
\usepackage[a5paper]{geometry}
\geometry{verbose,tmargin=1.5cm,bmargin=1.5cm,lmargin=2.5cm,rmargin=2.5cm}
\setcounter{secnumdepth}{-1}
\setcounter{tocdepth}{0}
\usepackage{varioref}
\usepackage{amsthm}
\usepackage{amsmath}
\usepackage{babel}
\usepackage{graphicx}

\begin{document}

\title{Sborník hádanek}
\author{Juda Kaleta}
\date{17. dubna 2010}

\maketitle

\tableofcontents{}


\chapter{Úvod}

Proč vlastně takovýto sborník všech možných i nemožných hádanek vytvářím?
Hlavním důvodem je to, že je všechny, chca nechca, zapomínám. Ne že
bych je zapomínal úplně, to ne, jen si na ně nemůžu vzpomenout.
A~myslím, že napsat si je, je ta nejspolehlivější cesta jak tomu zabránit.


\subsection{Co zde najdete?}

Spoustu hádanek. Opravdu mne baví je lidem dávat, protože podle toho,
jakým způsobem je řeší, se pozná mnoho z~lidského charakteru a osobnosti.
Od typů lidí, kteří po deseti vteřinách lehkého zamyšlení se, chtějí
znát řešení, po typy které dokáží přemýšlet deset minut s~různorodým
výsledkem, k~lidem, kteří nad některými hádankami stráví bezproblémů
dva dni vytrvalého přemýšlení - ano, všechny takové znám a je to jen
to nejpovrchnější třídění. Na každé hádance se odkryje něco jiného
- zbrklost, předsudky, nelogičnost uvažování.

Většina z~lidí má ráda hádanky. Možná si každý z~nás má potřebu něco
dokazovat, nebo se jen to forma relaxace, kdy přijdete na jiné myšlenky.
Existují ale i lidé, kteří nemají potřebu hádanky řešit. Některým
už mozek zlenivěl, jiní tvrdí, že mají i svých starostí dost a ten
zbytek jsou lidé tak praktičtí, že nevidí důvod, proč by hádanku měli
řešit (ledaže by za to bylo něco víc než dobrý pocit). Je jim to na
škodu, na spoustu hádanek sice nepřijdete, ale pokud se donutíte zamyslet,
je vám to ku prospěchu. Rozvíjíte si logické myšlení.

Mnoho z~hádanek souvisí s~opravdovým životem. Jako krásný příklad
můžu uvést hádanku ,,Pytlík bonbónů{}``. Návod na řešení používám
v~praxi už léta, osvědčil se. Někdy zase řešení použitelné nejsou,
buď jsou situace velmi absurdní, nebo jsou to situace do kterých se
nemáte šanci dostat - anebo a to dost často, proč by to normální člověk
vůbec řešil.


\subsection{Co zde nenajdete?}

Tady je odpověď velmi jednoduchá - hádanky, které se mi nelíbí, nebo
sem nepatří. Ty které sem nepatří jsou hádanky typu ,,leze, leze,
po železe...{}`` nebo ,,čtyři rohy, čtyři nohy, co je to?{}```.
Nejsou to totiž hádanky, u~kterých byste měli valnou šanci přijít
na řešení, pokud už jste ji někde neslyšeli. Pokud po nich toužíte,
knížek naleznete dost, nebo si zajděte za Glumem%
\footnote{Postava z~knihy J. R. Tolkiena v~knížce Hobit (objevuje se i v~Pánovi
Prstenů). Utká se v~,,hádankovém souboji{}`` s~hobitem Bilbem Pytlíkem.%
}.

Pak jsou to hádanky, z~kterých je nejznámější takzvaná Einsteinova%
\footnote{http://cs.wikipedia.org/wiki/Einsteinova\_hádanka%
}. Jsou to hádanky o~zadaných informacích a kombinaci. Na kteroukoliv
z~nich přijdete během půl hodiny, na většinu rychleji. Einstein sice
tvrdil, že tu jeho dokáží vyřešit pouze 2 \% lidí na planetě, je ale
jasné, že v~tomto se mýlil. Samozřejmě existuje výjimka potvrzující
pravidlo, i zde jednu podobného typu naleznete. Pochází ze soutěže Pikomat%
\footnote{http://pikomat.mff.cuni.cz}, i když bude lehce modifikovaná -
její název je ,,Tři cesty``.


\subsection{Obtížnost hádanek}

Je velmi těžké posoudit obtížnost té určité hádanky, přestože jsem
každou z~nich pokládal mnoho lidem. Jsou hádanky typu ,,Město mudrců{}``,
které vyřeší opravdu jen málo kdo, snad jen Hercule Poirot, je zde
potřeba přijít na geniální kombinace a vydedukovat si princip od nuly.
Pak existují hádanky s~více možnými řešeními. Ano, některé hádanky
mají víc řešení, pokud budu znát všechny elegantní, napíši je, ty
kostrbaté vynechávám. Obtížnost budu proto udávat procentuálně, odhadem.
Přiznávám se, že na mnohá řešení jsem nepřišel. Někdy mi to nedalo
a kouknul jsem se ,,jak to má být{}`` po pár minutách, jindy jsem
přiznal, že na to nemám. Nepřišel jsem na mnoho hádanek a to zejména
proto, že jsem k~tomu neměl motivaci. Pokud si hádanku sám přečtete
a máte po ruce řešení, neodoláte. Mnohem lepší je pracovat v~týmu,
alespoň ve dvou, pak je motivace myslet mnohem, mnohem vyšší.


\subsection{Odkud hádanky pocházejí?}

Snad žádnou z~hádanek jsem nevymyslel já. Nevím, kdo má mozek na to
je vymýšlet, nevím jestli vznikají postupně, či nápadem v~dané životní
situaci. Nemůžu proto ani pořádně a svědomitě uvádět zdroje, ze kterých čerpám.
Není nezvyklé, aby se jedna a ta samá hádanka objevovala na třech různých webech,
některé dokonce i ve více jazycích. Kdybyste proto někdo narazil na
hádanku, která je prokazatelně vaše, nebojte se mi ozvat
\footnote{Třeba na mail \emph{juda.kaleta@gmail.com}}. Pokud budu přejímat od
jiných autorů, budu se je snažit citovat, většinou to bohužel nelze.

%CONTENT%

%SOLUTIONS%

\end{document}
